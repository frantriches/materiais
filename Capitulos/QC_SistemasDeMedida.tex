\section{Questões Medida de Tempo}

\begin{enumerate}[1)]
 \item (UFSCAR - 2017) Três irmãs, Vânia, Beth e Raquel, moram em cidades diferentes, mas próximas da cidade de seus pais. Elas combinaram de visitá-los no mesmo dia. Neste dia, Vânia saiu de sua casa às 8h, Beth saiu de casa às 9h, Raquel saiu 40 minutos depois de Vânia e chegou à casa de seus pais às 9h30min. A viagem de Beth demorou uma vez e meia o tempo despendido por Raquel em sua viagem. A que horas Beth chegou à casa de seus pais?
 \begin{enumerate}[a)]
 \item 10h
 \item 9h45min
 \item 10h15min
 \item 10h30min
 \item 10h45min
 \end{enumerate}

 \item (VUNESP - 2017) Uma palestra teve início às 9 horas e 30 minutos e, após 1 hora e 5 minutos, sofreu uma interrupção de 10 minutos, retomando, em seguida, por mais 35 minutos até o intervalo, que durou 25 minutos. Após o intervalo, a palestra continuou por 1 hora e 20 minutos, chegando, assim, ao seu término, que ocorreu às:
 \begin{enumerate}[a)]
 \item 13h10min
 \item 13h05min
 \item 12h50min
 \item 12h35min
 \item 12h20min
 \end{enumerate}

 \item (RBO - 2017) Supondo que o relógio da estação Barueri atrase 23 segundos a cada 7 horas, e que o mesmo mantenha essas mesmas condições, então, em 7 dias, esse relógio atrasará:
 \begin{enumerate}[a)]
 \item 8 minutos e 24 segundos
 \item 8 minutos e 54 segundos
 \item 9 minutos e 2 segundos
 \item 9 minutos e 12 segundos
 \item 9 minutos e 20 segundos
 \end{enumerate}

 \item (VUNESP - 2017) Um agente de saúde está auxiliando no atendimento telefônico para dar esclarecimentos sobre uma determinada campanha de vacinação. Se cada atendimento tem duração média de 5 minutos, em 3 horas e 30 minutos de trabalho contínuo, o agente irá realizar uma quantidade de atendimentos igual a:
 \begin{enumerate}
 \item 34.
 \item 36.
 \item 38.
 \item 40.
 \item 42.
 \end{enumerate}

 \item (VUNESP - 2017) Renata foi realizar exames médicos em uma clínica. Ela saiu de sua casa às 14h 45 min e voltou às 17h 15 min. Se ela ficou durante uma hora e meia na clínica, então o tempo gasto no trânsito, no trajeto de ida e volta, foi igual a:
 \begin{enumerate}
 \item 1/2 h.
 \item 3/4 h.
 \item 1 h.
 \item 1h 15min.
 \item 1 1/2h.
 \end{enumerate}

 \item (MS Concursos - 2017) João estuda à noite e sua aula começa às 18h40min. Cada aula tem duração de 45 minutos, e o intervalo dura 15 minutos. Sabendo-se que nessa escola há 5 aulas e 1 intervalo diariamente, pode-se afirmar que o término das aulas de João se dá às:
 \begin{enumerate}
 \item 22h30min.
 \item 22h40min.
 \item 22h50min.
 \item 23h.
 \end{enumerate}

 \item (NC-UFPR - 2016) Quantos segundos tem uma semana?
 \begin{enumerate}
 \item Mais de 5.000 e menos de 8.000.
 \item Mais de 10.000 e menos de 20.000.
 \item Mais de 45.000 e menos de 100.000.
 \item Mais de 200.000 e menos de 400.000.
 \item Mais de 600.000 e menos de 1.000.000.
 \end{enumerate}

 \item (TJ/SC - 2018) Em sua empresa, quando Hugo trabalha além do tempo regulamentar, esse tempo extra é computado e acumulado em minutos. No fim do mês, somente os números inteiros de horas extras trabalhadas são pagas na razão de R\$54,00 por hora.

  No mês de maio, Hugo trabalhou, além do tempo regulamentar, por 500 minutos.

  O valor que Hugo recebeu a mais pelas horas extras foi de:
  \begin{enumerate}
  \item R\$ 324,00
  \item R\$ 378,00
  \item R\$ 432,00 (*)
  \item R\$ 450,00
  \item R\$ 486,00
 \end{enumerate}

\end{enumerate}

Gabarito: 1 c); 2 b); 3 d); 4 e); 5 c); 6 b); 7 e).

\section{Questões Medidas de Comprimento}
\begin{enumerate}
 \item (FUNDEP - 2017) João participou da última edição da Volta Internacional da Pampulha, uma das grandes provas do calendário brasileiro, realizada no primeiro domingo de dezembro em Belo Horizonte. O percurso total dessa prova é de 17,8 km. João conseguiu percorrer 9,75 km da prova.

Quantos quilômetros faltaram para ele concluir o percurso
 \begin{enumerate}
 \item 28,55 km
 \item 17,80 km
 \item 9,75 km
 \item 8,05 km
\end{enumerate}

\item (MS Concursos) Um quilômetro pode ser totalmente dividido em:
\begin{enumerate}
 \item 100 partes de 1 decímetro.
 \item 10 partes de 10 decâmetros.
 \item 1000 partes de 10 metros.
 \item 1000 partes de 1 centímetro.
\end{enumerate}

\item (INAZ do Pará - 2016) A Grande Muralha da China, com 7,8 metros de altura, começou a ser construída em 215 a.C. e foi erguida para proteger a região da invasão de nômades vindos do norte. A conversão da medida dessa altura foi realizada corretamente em:
\begin{enumerate}
 \item 0,078 cm.
 \item 0,78 dm.
 \item 0,078 hm.
 \item 0,78 km.
 \item 0,0078 mm.
\end{enumerate}

\item (VUNESP - 2017) Um ciclista venceu uma competição percorrendo 5 km em 14 minutos e 36 segundos. O 2° colocado chegou 27 segundos depois do primeiro, e o 3° colocado chegou 1 minuto e 15 segundos depois do 2° colocado. O tempo em que o 3° colocado demorou para percorrer os 5 km foi:
\begin{enumerate}
 \item 14 min e 27 s.
 \item 14 min e 51 s.
 \item 15 min e 27 s.
 \item 16 min e 18 s.
 \item 16 min e 36 s.
\end{enumerate}

 \item (Sociesc - 2009) Rosa quer dividir uma fita de 1,15 m de comprimento em dois pedaços, de maneira que um dos pedaços tenha 18 cm a mais do que o outro. O comprimento do pedaço maior é:
 \begin{enumerate}
  \item 75,5 cm
  \item 45,6 cm
  \item 64,6 cm
  \item 66,5 cm
 \end{enumerate}

 \end{enumerate}

 Gabarito: 1 d); 2 b); 3 b); 4 d); 5 d).

 \section{Questões Medidas de Superfície}
\begin{enumerate}
 \item (RBO - 2017) Para ladrilhar uma sala quadrada, foram utilizadas exatamente 400 peças de cerâmica cujas medidas das arestas são iguais. O perímetro de cada cerâmica é igual a 8,4 decímetros, então, a área da sala, em metros quadrados, é igual a:
 \begin{enumerate}
 \item 12,96.
 \item 14,44.
 \item 16,00.
 \item 16,81.
 \item 17,64.
\end{enumerate}

\item (IDHT - 2016) Um hectare de terra pode ser pensado como um quadrado com 100 metros de lado. Com essa visão intuitiva de um hectare, qual seria o seu perímetro?
\begin{enumerate}
 \item 100 m
 \item 200 m
 \item 300 m
 \item 400 m
 \item 500 m
\end{enumerate}

\item (KLC - 2016) Pedro tem uma chácara de $2500 m^2$. Assinale a alternativa correta para o número de hectares que esta chácara tem.
\begin{enumerate}
 \item 25 hectares.
 \item 250 hectares.
 \item 0,25 hectares.
 \item 2,5 hectares.
 \item Nenhuma alternativa está correta (NDA).
\end{enumerate}
\end{enumerate}

Gabarito: 1 e); 2 d); 3 c).


\section{Questões Medidas de Volume e Capacidade}

\begin{enumerate}[1)]
 \item (VUNESP - 2017) Uma indústria produz regularmente 4500 litros de suco por dia. Sabe-se que a terça parte da produção diária é distribuída em caixinhas P, que recebem 300 mililitros de suco cada uma. Nessas condições, é correto afirmar que a cada cinco dias a indústria utiliza uma quantidade de caixinhas P igual a:
 \begin{enumerate}
 \item 25000.
 \item 24500.
 \item 23000.
 \item 22000.
 \item 20500.
 \end{enumerate}

 \item (VUNESP - 2017) Com 150 litros de uma matéria-prima concentrada, são feitos 350 litros de um determinado produto A. Sabendo-se que essa matéria-prima é comprada ao valor $R\$ 12,50$ o litro, e que um litro do produto A é comercializado por $R\$ 7,00$, para se obter uma receita de exatamente $R\$ 5.880,00$ com a venda do produto A, o fabricante deste produto gastará, com a referida matéria­-prima, o valor exato de:
 \begin{enumerate}
 \item $R\$ 4.100,00$.
 \item $R\$ 4.300,00$.
 \item $R\$ 4.500,00$.
 \item $R\$ 4.700,00$.
 \item $R\$ 4.900,00$.
 \end{enumerate}

 \item (CS-UFG - 2016) Um determinado tipo de óleo é armazenado num tanque cúbico que tem capacidade de $4 096$ litros. Sabendo-se que 1 litro equivale a $10^3 cm^3$ , a medida da aresta do recipiente, em metros, é:
\begin{enumerate}
 \item 0,16
 \item 1,6.
 \item 16.
 \item 160.
\end{enumerate}

 \item (IDHTEC - 2016) Quantos copos de $200 cm^3$ são necessários para esvaziar totalmente um barril com 50 litros de vinho?
 \begin{enumerate}
 \item 25000.
 \item 2500.
 \item 250.
 \item 25.
 \item 2,5.
\end{enumerate}

\item (OBJETIVA - 2016) Quantas caixas d’água de 500 litros, serão necessárias, no mínimo, para encher completamente uma piscina de $12m^3$?
\begin{enumerate}
 \item 12
 \item 20
 \item 24
 \item 32
\end{enumerate}

 \item (Sociesc - 2009) Uma vinícola vai engarrafar o vinho que tem armazenado em 20 barris, com 120 litros cada um. Vai engarrafá-los em garrafas que contém 750 ml cada. O número de garrafas necessárias para engarrafar todo o vinho armazenado é:
  \begin{enumerate}
  \item 3.200
  \item 3.500
  \item 4.000
  \item 2.500
 \end{enumerate}

  \item (Sociesc - 2009) Lucas encheu o tanque de seu carro, que comporta 60 litros de combustível, e saiu para fazer uma viagem de negócios. No primeiro dia, gastou 13,4 litros, no dia seguinte, 9,7 litros e no terceiro dia, 16,4 litros. Sendo assim, ao final do terceiro dia ainda restam no tanque de combustível do seu carro:
  \begin{enumerate}
  \item 20,5 litros
  \item 10,5 litros
  \item 30,5 litros
  \item 15,5 litros
 \end{enumerate}

\end{enumerate}

 Gabarito: 1 a); 2 c); 3 b); 4 c); 5 c); 6 a); 7 a).

 \section{Questões Sistema Monetário}

\begin{enumerate}[1)]
 \item (FCC - 2011) No Brasil, o sistema monetário adotado é o decimal. Por exemplo:

$205,42 \text{ reais} = (2 \times 10^2 + 0 \times 10^1 + 5 \times 10^0 + 4 \times 10^{-1} + 2 \times 10^{-2})$ reais. Suponha que em certo país, em que a moeda vigente é o “mumu”, o sistema monetário seja binário. O exemplo seguinte mostra como converter certa quantia, dada em “mumus”, para reais:

$110,01 \text{ mumus} = (1 \times 2^2 + 1 \times 2^1 + 0 \times 2^0 + 0 \times 2^{-1} + 1 \times 2^{-2}) \text{ reais} = 6,25$ reais Com base nessas informações, se um brasileiro em viagem a esse país quiser converter $385,50$ reais para a moeda local, a quantia que ele receberá, em “mumus”, é:

\begin{enumerate}
\item 10 100 001,11.
\item 110 000 001,1.
\item 110 000 011,11.
\item 110 000 111,1.
\item 111 000 001,11.
\end{enumerate}

\item (MPE-GO - 2016) Há 22 anos, em 1º de julho de 1994, entrava em vigor o real, moeda que pôs fim à hiperinflação que assolava a população brasileira. Nesse novo sistema monetário, cada real valia uma URV (Unidade Real de Valor), que, por sua vez, valia 2750 cruzeiros reais. Dessa forma, 33550 cruzeiros reais valiam:
\begin{enumerate}
\item 10,50 URV.
\item 11,70 URV.
\item 12,50 URV.
\item 12,20 URV.
\item 13,70 URV
\end{enumerate}

\end{enumerate}

Gabarito: 1 b); 2 d).

\section{Questões Medida de Massa}
\begin{enumerate}
 \item (RBO - 2017) Uma garrafa cheia de suco de uva pesa 1450 gramas. A mesma garrafa contendo um terço de suco de uva pesa 810 gramas. Logo, essa garrafa com metade da capacidade desse mesmo suco pesará:
 \begin{enumerate}
 \item 850 gramas
 \item 900 gramas
 \item 920 gramas
 \item 970 gramas
 \item 980 gramas
\end{enumerate}


\item (UniRV-GO - 2017) Uma empresa farmacêutica distribuiu 14400 litros de uma substância líquida em recipientes de $72 cm^3$ cada um. Sabe-se que cada recipiente, depois de cheio, tem 80 gramas. A quantidade de toneladas que representa todos os recipientes cheios com essa substância é de:
\begin{enumerate}
 \item 16.
 \item 160.
 \item 1600.
 \item 16000.
\end{enumerate}

\item (IESES - 2017) Ao somarmos 72,5 decigramas com 0,875 decagramas teremos?
\begin{enumerate}
 \item 7,3375 gramas.
 \item 73,375 gramas.
 \item 9,475 gramas.
 \item 16 gramas.
\end{enumerate}

\item (CETREDE - 2016) Comprei um aquário para meu filho com capacidade de $72.000 cm^3$. Quantos quilogramas o aquário pesará depois que estiver cheio d’água se, vazio, ele pesa 3 kg?
\begin{enumerate}
 \item 7,2 kg
 \item 72 kg
 \item 7,5 kg
 \item 76 kg
 \item 75 kg
\end{enumerate}


\end{enumerate}

Gabarito: 1 d); 2 a); 3 d); 4 e).

\section{Questões Gerais}

\begin{enumerate}
 \item (UFPE/Covest - 2015) A empresa gestora de um porto precisa construir um novo cais. A laje de betão para o cais, na forma de um paralelepípedo retângulo, precisa ter 75 m de comprimento, 60 m de largura e 0,3 m de espessura. Se a carga de um caminhão cheio de betão é de $25 m^3$ , quantos caminhões carregados de betão serão necessários para construir a laje? Dado: o volume de um paralelepípedo retângulo é dado pelo produto das medidas de seu comprimento, largura e espessura.
 \begin{enumerate}
 \item 50
 \item 51
 \item 52
 \item 53
 \item 54
\end{enumerate}

 \item (UFPE/Covest - 2015) Uma determinada tarefa é executada pelo funcionário X, em 4 horas, e pelo funcionário Y em 6 horas, trabalhando sozinhos. Certo dia, os dois funcionários começaram a executar a tarefa juntos, às 08h00min. Às 09h00min, o funcionário Y precisou se ausentar pelo período de uma hora e, depois, voltou a executar a tarefa. A que horas a tarefa foi concluída pelos dois funcionários?
 \begin{enumerate}
 \item 10h36min
 \item 10h40min
 \item 10h44min
 \item 10h48min
 \item 10h52min
\end{enumerate}

 \item (UFPE/Covest - 2015) Dois irmãos trabalham na loja de sua família, que se situa a 8 Km da residência deles. Um dos irmãos trabalha no turno da manhã, e o outro, no turno da tarde. Diariamente, eles percorrem o mesmo trajeto, e se encontram no caminho de casa, para que um entregue ao outro a chave da loja. Um dos irmãos sai de casa às 12h00min e demora 10 minutos para percorrer cada quilômetro, enquanto o outro sai da loja no mesmo momento e demora 15 minutos para percorrer cada quilômetro. A que horas os dois irmãos se encontram?
 \begin{enumerate}
 \item 12h44min
 \item 12h45min
 \item 12h46min
 \item 12h47min
 \item 12h48min
\end{enumerate}

\item (UFPE/Covest - 2015) Uma indústria farmacêutica fabrica 2.600 litros de uma vacina que devem ser colocados em ampolas de $25 cm^3$ cada uma. Quantas ampolas serão obtidas com essa quantidade de vacina?
\begin{enumerate}
\item 104
\item 1.040
\item 10.400
\item 104.000
\item 1040.000
\end{enumerate}
\end{enumerate}
Gabarito: 1 e); 2 d); 3 e); 4 d).
