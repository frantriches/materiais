\chapter{Estatística}

 {\color{red} média aritmética, média ponderada, moda}

Sempre que fazemos uma pesquisa de levantamento de dados, seja uma pesquisa eleitoral, uma pesquisa para determinar as características dos consumidores de um determinado produto, ou taxa de natalidade ou mortalidade no Brasil, dentre outras, precisamos de conhecimentos básicos e avançados de estatística para garantir que a pesquisa retrate a realidade. Precisamos também destes conhecimentos para entender corretamente os gráficos e tabelas utilizados para representar os dados obtidos através da pesquisa, e conseguir extrair destas ferramentas o resultado da pesquisa.

Vamos aqui tratar de alguns tópicos da teoria estatística considerados a base de toda a estatística, vale ressaltar que não passaremos nem perto de cobrir toda a abrangência desta ciência. Como ponto de partida de nossas estudos vamos pensar nas pesquisas do tipo \textit{Pesquisa de Opinião}, por ser um exemplo de pesquisa de levantamento de dados no qual podemos aplicar toda a teoria que iremos estudar, e também um modelo mais fácil de criar exemplos do dia-a-dia.

Considerando então uma pesquisa de opinião, sempre que pensamos em uma pesquisa de opinião, após termos o tema de nossa pesquisa em mãos precisamos pensar quem são as pessoas que devemos entrevistar para obter os dados da pesquisa, por exemplo, se nossa pesquisa é sobre candidatos a presidente do Brasil, devemos perguntar a opinião de eleitores brasileiros, ou seja, brasileiros com 16 anos ou mais, neste caso a opinião de crianças de 10 anos não é interessante para demonstrar a intenção de votos dos brasileiros. Mas se a pesquisa tem por objetivo descobrir qual é o tipo de literatura que as crianças com 10 anos gostam de ler, então precisamos perguntar para as crianças com 10 anos, fazer esta pesquisa com pessoas com mais de 10 anos não irá nos dar o resultado buscado.

Neste dois casos, os brasileiros com 16 anos ou mais, e as crianças de 10 anos representam o que chamamos na estatística de \textbf{Universo estatístico ou População estatística}. De maneira geral:

 \vskip0.3cm
 \colorbox{azul}{
 \begin{minipage}{13cm}
 \begin{center}
  O \textbf{Universo estatístico ou População estatística} é o conjunto formado por todos os elementos que possam oferecer dados pertinentes ao assunto em questão. No caso particular das pesquisas de opinião, é o conjunto formado por todas as pessoas, das quais a opinião sobre o tema em questão nos interessa.
 \end{center}
 \end{minipage}}
 \vskip0.3cm

 Como muitas vezes o universo ou população para uma determinada pesquisa é muito grande, e acaba ficando inviável coletar dados de todos os elementos, nestes casos escolhemos então uma \textbf{amostra} da população. Portanto,

  \vskip0.3cm
 \colorbox{azul}{
 \begin{minipage}{13cm}
 \begin{center}
  Uma \textbf{amostra} é um subconjunto da população, do qual iremos coletar dados de todos os elementos.
 \end{center}
 \end{minipage}}
 \vskip0.3cm

 A amostra é escolhida aleatoriamente garantindo que o resultado da pesquisa obtido através dos dados coletados de cada um dos elementos da amostra, seja equivalente ao que iríamos obter que se coletássemos dados de toda a população.

 Após a coleta de dados feita, os dados são organizados em \textbf{tabelas} possibilitando uma leitura clara e objetiva do resultado da pesquisa.

 A partir das tabelas construídas podemos gerar \textbf{gráficos} para representar os dados tabulados, eles são muito utilizados pois dão ao leitor uma impressão mais rápida do resultado obtido na pesquisa. Um gráfico deve ser simples, claro e expressar a verdade sobre os dados coletados.


   \vskip0.3cm
 \colorbox{amarelo}{
 \begin{minipage}{13cm}
 \begin{center}
  As \textbf{tabelas} e os \textbf{gráficos} devem conter título e fonte. O título informa ao leitor o que eles apresentam, além de quando e onde o fato ocorreu. A fonte deve identificar a origem das informações.
 \end{center}
 \end{minipage}}
 \vskip0.3cm
