
\section{Questões}
\begin{enumerate}
 \item (SOCIESC - Téc. Enfermagem) Assinale a alternativa que apresenta o $75\degree$ termo da PA $(2, 5, \cdots)$:
  \begin{enumerate}
  \item 222
  \item 223
  \item 224
  \item 225
  \item 226
 \end{enumerate}

 \item (Lógica - Fundatec - 2018) Um maratonista em treinamento corre todos os dias da semana 700 metros a mais do que no dia anterior. Após 14 dias de treinamento, ele correu um percurso total de 93.100m. A distância percorrida, em metros, no sexto dia foi de:

\begin{enumerate}[a)]
\item 2.100
\item 4.200
\item 5.600
\item 6.300
\item 6.650
\end{enumerate}


\item (Lógica - Fundatec - 2018) Considere as seguintes sequências de caracteres:

1ª sequência: \&\&\&

2ª sequência: \&\&\&\&\&\&

3ª sequência: \&\&\&\&\&\&\&\&\&\&\&\&

4ª sequência: \&\&\&\&\&\&\&\&\&\&\&\&\&\&\&\&\&\&\&\&\&\&\&\&

A quantidade de caracteres na décima segunda sequência é:

\begin{enumerate}[a)]
\item 192
\item 384
\item 768
\item 3.072
\item 6.144
\end{enumerate}


\item (Covest - UFPE - 2013) Um botijão de gás de cozinha cheio contém 13 kg de gás. No domicílio X, consome-se 0,5 kg de gás por dia e no domicílio Y consome-se 0,3 kg. No início de certo dia, no domicílio X, o botijão está cheio, enquanto no domicílio Y, já foram gastos 3 kg. Depois de quantos dias as quantidades de gás nos botijões dos dois domicílios serão iguais?
\begin{enumerate}
\item 11
\item 12
\item 13
\item 15
\item 16
\end{enumerate}

\item (Covest - UFPE - 2013) João atualiza o antivírus do seu computador a cada 22 dias e Maria atualiza o antivírus a cada 25 dias. Em certa segunda-feira, os dois atualizaram o antivírus no mesmo dia. Na próxima vez em que os dois atualizarem o antivírus no mesmo dia, qual será o dia da semana?
\begin{enumerate}
\item Segunda-feira
\item Terça-feira
\item Quarta-feira
\item Quinta-feira
\item Sexta-feira
\end{enumerate}

\item (FGV- 2018)
Marta tem 20 bolas numeradas de 1 a 20. Ela pinta de vermelho todas as bolas cujo número é múltiplo de 4, isto é, 4, 8, 12 etc.
A seguir, ela pinta de azul as bolas cujos números são antecessores de números das bolas que foram pintadas de vermelho.
Por último, ela pinta de verde as bolas cujos números são sucessores de números das bolas que foram pintadas de vermelho.
Nenhuma outra bola foi pintada.
O número de bolas não pintadas é
\begin{enumerate}
\item 4
\item 5
\item 6
\item 7
\item 8
\end{enumerate}

\item (CESPE - 2018) Se, em uma progressão aritmética, o segundo termo for igual a 1 e o quinto termo for igual a 11, então o décimo primeiro termo será igual a
\begin{enumerate}
\item 30
\item 31
\item 35
\item 50
\item 95
\end{enumerate}

\item (UTFPR - 2018) Viviane iniciou a leitura de um livro com 538 páginas. No primeiro dia, ela leu 5 páginas, no segundo, ela leu duas páginas a mais que no primeiro dia. E assim por diante, a cada dia ela leu duas páginas a mais que no dia anterior. Assinale, após 19 dias de leitura, quantas páginas ainda faltam para ela ler.
\begin{enumerate}
\item 101
\item 41
\item 207
\item 437
\item 311
\end{enumerate}

\item (UFOP - 2018) Três termos consecutivos de uma progressão geométrica crescente são $x$, $x+20$ e $2x + 10$. A razão dessa progressão é:
\begin{enumerate}
\item 40
\item 3/2
\item 1/2
\item -10
\end{enumerate}



\end{enumerate}

Gabarito: 1 c); 2 c); 3 e); 4 d); 5 d); 6 c); 7 b); 8 a); 9 b).
