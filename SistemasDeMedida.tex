\chapter{Sistemas De Medida}
%rever questões e colocar em ordem de uso, e unificar a ``cara'' das tabelas.

\section{Medida de tempo}
A unidade padrão para medir tempo de acordo com o Sistema Internacional (SI) é o segundo $(s)$. Cujos múltiplos e submúltiplos são dados pela tabela abaixo:
 \begin{table}[H]
  \centering
  \begin{tabular}{|c|c|c|c|} \hline
  \rowcolor{cinza}
  \multicolumn{1}{|c|}{\textbf{Unid. Fund.}} & \multicolumn{3}{|c|}{\textbf{Múltiplos}} \\ \hline
  segundo & minuto & hora & dia \\ \hline
  s & min & h & d \\ \hline
  $1 s$ & $1 min = 60 s$ & $1 h = 60 min$ & $1 d= 24 h$ \\ \hline
  \end{tabular}
 \end{table}

Além destas unidades de tempo, podemos também relacionar as seguintes unidades:
\begin{table}[H]
\centering
\begin{tabular}{|c|c|} \hline
\rowcolor{cinza}
 \textbf{Unidade} & \textbf{Equivale à} \\ \hline
 Semana & 7 dias \\ \hline
 Quizena & 15 dias \\ \hline
 Mês & 30 dias* \\ \hline
 Bimestre & 2 meses \\ \hline
 Trimestre & 3 meses \\ \hline
 Quadrimestre & 4 meses \\ \hline
 Semestre & 6 meses \\ \hline
 Ano & 12 meses \\ \hline
 Década & 10 anos \\ \hline
 Século & 100 anos \\ \hline
 Milênio & 1000 anos \\ \hline
\end{tabular}
\end{table}

 *O mês comercial utilizado em cálculos financeiros possui por convenção 30 dias.
\newpage

\newpage
\section{Sistema métrico decimal}

\subsection{Medidas de comprimento}
A unidade padrão para medir comprimentos de acordo com o Sistema Internacional (SI) é o metro $(m)$. Cujos múltiplos e submúltiplos são dados pela tabela abaixo:

 \begin{table}[H]
 \centering
 \begin{tabular}{|c|c|c|c|c|c|c|} \hline
 \rowcolor{cinza}
  \multicolumn{3}{|c|}{\textbf{Múltiplos}}
 & \multicolumn{1}{|c|}{\textbf{Unid. Fund.}} & \multicolumn{3}{|c|}{\textbf{Submúltiplos}} \\ \hline
 quilômetro & hectômetro & decâmetro & metro & decímetro & centímetro & milímetro \\ \hline
 $km$ & $hm$ & $dam$ & $m$ & $dm$ & $cm$ & $mm$ \\ \hline
 $1000 m$ & $100 m$  & $10 m$ & $1 m$ & $0,1 m$ & $0,01 m$ & $0,001 m$ \\ \hline
 $10^3 m$ & $10^2 m$ & $10^1 m$ & $1 m$ & $10^{-1} m$ & $10^{-2} m$ & $10^{-3} m$ \\ \hline
 \end{tabular}
 \end{table}

 Existem outras unidades de medida mas que não pertencem ao sistema métrico decimal. Vejamos as relações entre algumas dessas unidades e as do sistema métrico decimal:

1 polegada = 25,4 milímetros

1 milha = 1 609 metros

1 légua = 5 555 metros

1 pé = 30 centímetros

Obs: valores aproximados.



\subsection{Medidas de superfície}
A unidade padrão para medir superfície de acordo com o Sistema Internacional (SI) é o metro quadrado $(m^2)$. Cujos múltiplos e submúltiplos são dados pela tabela abaixo:

 \begin{table}[H]
 \centering
 \begin{tabular}{|c|c|c|c|c|c|c|} \hline
 \rowcolor{cinza}
  \multicolumn{3}{|c|}{\textbf{Múltiplos}}
 & \multicolumn{1}{|c|}{\textbf{Unid. Fund.}} & \multicolumn{3}{|c|}{\textbf{Submúltiplos}} \\
 \hline
 $km^2$ & $hm^2$ & $dam^2$ & $m^2$ & $dm^2$ & $cm^2$ & $mm^2$ \\
 \hline
 $1000 000 m^2$ & $10 000 m^2$ & $100 m^2$ & $1 m^2$ & $0,01 m^2$ & $0,0001 m^2$ & $0,000001 m^2$ \\ \hline
 $10^6 m^2$ & $10^4 m^2$ & $10^2 m^2$ & $1 m^2$ & $10^{-2} m^2$ & $10^{-4} m^2$ & $10^{-6} m^2$ \\ \hline
 \end{tabular}
 \end{table}

 Quando queremos medir grandes porções de terra (como sítios, fazendas etc.) usamos uma unidade agrária chamada hectare $(ha)$.

O hectare é a medida de superfície de um quadrado de $100 m$ de lado.

$1$ hectare $(ha)$ = $1 hm^2$ = $10 000 m^2$

Em alguns estados do Brasil, utiliza-se também uma unidade não legal chamada alqueire.

    1 alqueire mineiro é equivalente a $48 400 m^2$.

    1 alqueire paulista é equivalente a $24 200 m^2$.



\subsection{Medidas de volume}
A unidade padrão para medir volume de acordo com o Sistema Internacional (SI) é o metro cúbico $(m^3)$. Cujos múltiplos e submúltiplos são dados pela tabela abaixo:

 \begin{table}[H]
 \centering
 \begin{tabular}{|c|c|c|c|c|c|c|} \hline
 \rowcolor{cinza}
  \multicolumn{3}{|c|}{\textbf{Múltiplos}}
 & \multicolumn{1}{|c|}{\textbf{Unid. Fund.}} & \multicolumn{3}{|c|}{\textbf{Submúltiplos}} \\
 \hline
 $km^3$ & $hm^3$ & $dam^3$ & $m^3$ & $dm^3$ & $cm^3$ & $mm^3$ \\
 \hline
 $1000 000 000 m^3$ & $1000 000 m^3$ & $1000 m^3$ & $1 m^3$ & $0,001 m^3$ & $0,000001 m^3$ & $0,000000001 m^3$ \\ \hline
 $10^9 m^3$ & $10^6 m^3$ & $10^3 m^3$ & $1 m^3$ & $10^{-3} m^3$ & $10^{-6} m^3$ & $10^{-9} m^3$ \\ \hline
 \end{tabular}
\end{table}

\subsection{Medidas de capacidade}
A unidade padrão para medir capacidade de um sólido de acordo com o Sistema Internacional (SI) é o litro $(l)$.

De acordo com o Comitê Internacional de Pesos e Medidas, o litro é, aproximadamente, o volume equivalente a um decímetro cúbico, ou seja:

$1$ litro= $1,000027 dm^3$

Porém, na prática, definimos:
$1$ litro = $1 dm^3$.

Cujos múltiplos e submúltiplos são dados pela tabela abaixo:

 \begin{table}[H]
 \centering
 \begin{tabular}{|c|c|c|c|c|c|} \hline
 \rowcolor{cinza}
  \multicolumn{2}{|c|}{\textbf{Múltiplos}}
 & \multicolumn{1}{|c|}{\textbf{Unid. Fund.}} & \multicolumn{3}{|c|}{\textbf{Submúltiplos}} \\
 \hline
 hectolitro & decalitro & litro & decilitro & centilitro & mililitro \\
 \hline
 $hl$ & $dal$ & $l$ & $dl$ & $cl$ & $ml$ \\ \hline
 $100 l$ & $10 l$ & $1 l$ & $0,1 l$ & $0,01 l$ & $0,001 l$\\ \hline
 $10^2 l$ & $10^1 l$ & $1 l$ & $10^{-1} l$ & $10^{-2} l$ & $10^{-3} l$\\ \hline
 \end{tabular}
\end{table}

Obs: Não é usado nem consta da lei o quilolitro.



\section{Sistema monetário brasileiro}
O sistema monetário de um país é composto por regras e bancos comerciais e estatais responsáveis pela circulação da moeda. No Brasil a moeda vigente é o Real e o órgão responsável pela administração e produção de cédulas e moedas é o Banco Central.

Historicamente o Brasil teve as seguintes moedas:

 \begin{table}[H]
 \centering
 \begin{tabular}{|c|c|c|c|} \hline
 \rowcolor{cinza}
 \multicolumn{4}{|c|}{\textbf{Unidades do sistema monetário Brasileiro}} \\
 \hline
 \textbf{Unidade monetária} & \textbf{Período de vigência} & \textbf{Símbolo} & \textbf{Correspondência} \\
\hline
Real (plural= Réis) & Período colonial até 07/10/1833 & R & R 1\$2000= 1/8 ouro de 22K \\
\hline
Mil Réis & 08/10/1833 a 31/10/1942 & R\$ & Rs 2\$500= 1/8 de ouro de 22K \\
\hline
Cruzeiro & 01/11/1942 a 30/11/1964 & Cr\$ & Cr\$ 1,00= Rs 1\$000 \\
\hline
Cruzeiro & 01/12/1964 a 12/02/1967 & Cr\$ & Cr\$ 1= Cr\$ 1,00 \\
\hline
Cruzeiro Novo & 13/02/1967 a 14/05/1970 & NCr\$ & Cz\$ 1,00= Cr\$ 1.000 \\
\hline
Cruzado Novo & 16/01/1989 a 15/03/1990 & NCz\$ & NCz\$= Cz\$ 1.000,00 \\
\hline
Cruzeiro & 16/03/1990 a 31/07/1993 & Cr\$ & Cr\$ 1,00= NCz\$ 1,00 \\
\hline
Cruzeiro Real & 01/08/1993 a 30/06/1994 & CR\$ & CR\$ 1,00= Cr\$ 1.000,00 \\
\hline
Real (plural= Reais) & A partir de 01/07/1994 & R\$ & R\$ 1,00= CR\$2.750,00 \\
\hline

\end{tabular}
\caption{Moedas Brasileiras.}
\end{table}


\section{Medida de massa}
A unidade padrão para medir massa de um sólido de acordo com o Sistema Internacional (SI) é o quilograma $(kg)$. Cujo submúltiplo mais utilizado é o grama $(g)$, mas este não é o único. Vejamos na tabela abaixo todos os múltiplos e submúltiplos do quilograma:

 \begin{table}[H]
 \centering
 \begin{tabular}{|c|c|c|c|c|c|c|} \hline
 \rowcolor{cinza}
  \multicolumn{3}{|c|}{\textbf{Múltiplos}}
 & \multicolumn{1}{|c|}{\textbf{Unid. Fund.}} & \multicolumn{3}{|c|}{\textbf{Submúltiplos}} \\
 \hline
 tonelada & quilograma & decagrama & grama & decigrama & centigrama & miligrama \\
 \hline
 $t$ & $kg$ & $dag$ & $g$ & $dg$ & $cg$ & $mg$ \\ \hline
 $1000000 g$ & $1000 g$ & $10 g$ & $1 g$ & $0,1 g$ & $0,01 g$ & $0,001 g$ \\ \hline
 $10^6 g$ & $10^3 g$ & $10^1 g$ & $1 g$ & $10^{-1} g$ & $10^{-2} g$ & $10^{-3} g$ \\ \hline
 \end{tabular}
\end{table}
